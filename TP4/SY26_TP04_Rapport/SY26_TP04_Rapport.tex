\documentclass[a4paper, 12pt]{article}
\usepackage[utf8]{inputenc}  % Unicode
\usepackage[french]{babel}
\usepackage{geometry}
\usepackage{graphics}
\usepackage[pdftex]{graphicx, color}
\usepackage{pdfpages}
\DeclareGraphicsExtensions{.jpg,.png}
\pdfoutput=1
\usepackage[pdftex,
	bookmarks = true,           % Signets
	bookmarksnumbered = true,   % Signets numerotes
	pdfstartview = FitV,        % La page prend toute la hauteur
	colorlinks=true,
	citecolor=black,urlcolor=blue,linkcolor=black,
	pdfauthor={Auteur},
	pdftitle={Titre},
 	pdfsubject={Sujet},
%	pdfkeywords={},	% Besoin de keywords ?
	plainpages=false,
	pdfpagelabels,
	breaklinks=true,
   	hyperindex,
	linktocpage=true	% pour colorier seulement le numéros dans la TOC	
]{hyperref}
\usepackage{float}
\usepackage{listings}
\usepackage{alltt}
\renewcommand{\ttdefault}{txtt}

\lstset{basicstyle=\ttfamily,
escapeinside={||},
mathescape=true}
\setcounter{secnumdepth}{3}
\newcommand{\HRule}{\rule{\linewidth}{0.5mm}}

\newcommand*\styleC{\fontsize{9}{10pt}\selectfont }
\newcommand*\styleD{\fontsize{9}{10pt}\usefont{OT1}{pag}{m}{n}\selectfont }

\makeatletter
% on fixe le langage utilisé
\lstset{language=matlab}
\edef\Motscle{emph={\lst@keywords}}
\expandafter\lstset\expandafter{
}
\makeatother



\definecolor{Ggris}{rgb}{0.45,0.48,0.45}

\lstset{emphstyle=\ttfamily\color{blue}, % les mots réservés de matlab en bleu
basicstyle=\ttfamily\styleC, % 
keywordstyle=\ttfamily,
commentstyle=\color{Ggris}\styleD, % \styleD commentaire en gris
numberstyle=\tiny\color{black},
numbers=left,
numbersep=10pt,
lineskip=0.7pt,
showstringspaces=false}
%  % inclure le fichier source
\newcommand{\FSource}[1]{%
\lstinputlisting[texcl=true]{#1}
}

%%%%%%%%%
\textwidth=15cm
\textheight=21cm
%\hoffset=-2.5cm
\tolerance=9000
\hbadness=9000
\pretolerance=2500


\begin{document}
%\rmfamily

\begin{titlepage}
\begin{center}



\textsc{\Large Rapport de TP - SY26}\\[0.5cm]
\vspace{4cm}
% Title
\HRule \\[0.4cm]
{ \huge \bfseries TP04 - La compression JPEG \\[0.4cm] }

\HRule \\[1.5cm]

% Author and supervisor
\begin{minipage}{0.4\textwidth}
\begin{flushleft} \large
R\'emi \textsc{Burtin}
\end{flushleft}
\end{minipage}
\begin{minipage}{0.4\textwidth}
\begin{flushright} \large
Cyril \textsc{Fougeray}
\end{flushright}
\end{minipage}

\vspace{4cm}

{\large \today}



\vfill
% Bottom
\includegraphics[width=0.25\textwidth]{logo.jpg}\\[0.5cm]

\textsc{\LARGE Universit\'{e} de Technologie de Compi\`{e}gne}\\[1.5cm]


\end{center}
\end{titlepage}


%\begin{abstract} 
%\end{abstract} 

%{\bf Keywords:} \newline


\clearpage

\section{Introduction}

Le but de ce TP est de mettre en œuvre certaines étapes de l’algorithme de compression JPEG.

\section{Transformée en cosinus discrète (DCT)}

\subsection{Mise en œuvre de la fonction MyDCT}\label{}

Cette première fonction nous permet de calculer la transformée en cosinus discrète d'un bloc de taille 8x8.

Pour cela, nous utilisons les formules suivantes :\\

$D = X_mBY_m^T$ \\ 

avec $X_m(u,x) = \frac{1}{2}C(u) cos(\frac{(2x + 1)\pi u }{16})$ et $Y_m(v,y) = \frac{1}{2}C(v) cos(\frac{(2x + 1)\pi v }{16})$ \\

en prenant $C(0) = \frac{1}{\sqrt{2}}$ et $C(k)=1$ pour $k \in [1..7]$. \\

Les coefficient $u$ et $v$ varie de 0 à 7, donc les matrices $X_m$ et $Y_m$ ont 8 lignes. Par ailleurs, la matrice $B$ avec laquelle nous travaillons est une matrice 8x8. Ainsi, les coefficients $x$ et $y$ sont pris entre 1 et 8. On obtient donc $X_m = Y_m$. \\

Nous calculons les matrices $X_m$ et $Y_m$ en remplaçant les 4 variables $(u, v, x, y)$ dans les formules, nous multiplions les matrices $X_m$, $B$ et $Y_m$ et obtenons ainsi la matrice $D$, la transformée en cosinus discrète de $B$.\\

Le code de la fonction est en annexe \ref{dct_code}.

\subsubsection{R\'esultats}
\label{sec:Resultats}

Après avoir exécuté la fonction, nous nous rendons compte du résultats : 

\begin{alltt}
>> Bref=[	139 144 149 153 155 155 155 155;
					144 151 153 156 159 156 156 156;
					150 155 160 163 158 156 156 156;
					159 161 162 160 160 159 159 159;
					159 160 161 162 162 155 155 155;
					161 161 161 161 160 157 157 157;
					162 162 161 163 162 157 157 157;
					162 162 161 161 163 158 158 158];

>> BrefDCT = MyDCT(Bref)

BrefDCT =

   1.0e+03 *

 1.2596   -0.0010   -0.0121   -0.0052    0.0021   -0.0017   -0.0027    0.0013
-0.0226   -0.0175   -0.0062   -0.0032   -0.0029   -0.0001    0.0004   -0.0012
-0.0109   -0.0093   -0.0016    0.0015    0.0002   -0.0009   -0.0006   -0.0001
-0.0071   -0.0019    0.0002    0.0015    0.0009   -0.0001   -0.0000    0.0003
-0.0006   -0.0008    0.0015    0.0016   -0.0001   -0.0007    0.0006    0.0013
 0.0018   -0.0002    0.0016   -0.0003   -0.0008    0.0015    0.0010   -0.0010
-0.0013   -0.0004   -0.0003   -0.0015   -0.0005    0.0017    0.0011   -0.0008
-0.0026    0.0016   -0.0038   -0.0018    0.0019    0.0012   -0.0006   -0.0004
\end{alltt}

La matrice $BrefDCT$ correspond au résultat souhaité. \\



\section{Quantification}



\section{Parcours en zigzag}

\section{Codage}

\section{Codage/décodage d'une image}

\newpage

\section{Conclusion}



\clearpage

%
% ANNEXE
%
\appendix

\section{Codes source MATLAB}
\subsection{Transformée en cosinus discrète d'un bloc 8x8}\label{dct_code}

\FSource{../MyDCT.m}

\newpage


\end{document}