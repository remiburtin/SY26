\documentclass[a4paper, 12pt]{article}
\usepackage[utf8]{inputenc}  % Unicode
\usepackage[french]{babel}
\usepackage{geometry}
\usepackage{graphics}
\usepackage[pdftex]{graphicx, color}
\usepackage{pdfpages}
\DeclareGraphicsExtensions{.jpg,.png}
\pdfoutput=1
\usepackage[pdftex,
	bookmarks = true,           % Signets
	bookmarksnumbered = true,   % Signets numerotes
	pdfstartview = FitV,        % La page prend toute la hauteur
	colorlinks=true,
	citecolor=black,urlcolor=blue,linkcolor=black,
	pdfauthor={Auteur},
	pdftitle={Titre},
 	pdfsubject={Sujet},
%	pdfkeywords={},	% Besoin de keywords ?
	plainpages=false,
	pdfpagelabels,
	breaklinks=true,
   	hyperindex,
	linktocpage=true	% pour colorier seulement le numéros dans la TOC	
]{hyperref}
\usepackage{float}
\usepackage{listings}
\usepackage{alltt}
\usepackage{amsmath}
\renewcommand{\ttdefault}{txtt}

\lstset{basicstyle=\ttfamily,
escapeinside={||},
mathescape=true}
\setcounter{secnumdepth}{3}
\newcommand{\HRule}{\rule{\linewidth}{0.5mm}}

\newcommand*\styleC{\fontsize{9}{10pt}\selectfont }
\newcommand*\styleD{\fontsize{9}{10pt}\usefont{OT1}{pag}{m}{n}\selectfont }

\makeatletter
% on fixe le langage utilisé
\lstset{language=matlab}
\edef\Motscle{emph={\lst@keywords}}
\expandafter\lstset\expandafter{
}
\makeatother



\definecolor{Ggris}{rgb}{0.45,0.48,0.45}

\lstset{emphstyle=\ttfamily\color{blue}, % les mots réservés de matlab en bleu
basicstyle=\ttfamily\styleC, % 
keywordstyle=\ttfamily,
commentstyle=\color{Ggris}\styleD, % \styleD commentaire en gris
numberstyle=\tiny\color{black},
numbers=left,
numbersep=10pt,
lineskip=0.7pt,
showstringspaces=false}
%  % inclure le fichier source
\newcommand{\FSource}[1]{%
\lstinputlisting[texcl=true]{#1}
}

%%%%%%%%%
\textwidth=15cm
\textheight=21cm
%\hoffset=-2.5cm
\tolerance=9000
\hbadness=9000
\pretolerance=2500


\begin{document}
%\rmfamily

\begin{titlepage}
\begin{center}



\textsc{\Large Rapport de TP - SY26}\\[0.5cm]
\vspace{4cm}
% Title
\HRule \\[0.4cm]
{ \huge \bfseries TP04 - La compression JPEG \\[0.4cm] }

\HRule \\[1.5cm]

% Author and supervisor
\begin{minipage}{0.4\textwidth}
\begin{flushleft} \large
R\'emi \textsc{Burtin}
\end{flushleft}
\end{minipage}
\begin{minipage}{0.4\textwidth}
\begin{flushright} \large
Cyril \textsc{Fougeray}
\end{flushright}
\end{minipage}

\vspace{4cm}

{\large \today}



\vfill
% Bottom
\includegraphics[width=0.25\textwidth]{logo.jpg}\\[0.5cm]

\textsc{\LARGE Universit\'{e} de Technologie de Compi\`{e}gne}\\[1.5cm]


\end{center}
\end{titlepage}


%\begin{abstract} 
%\end{abstract} 

%{\bf Keywords:} \newline


\clearpage

\section{Introduction}

L’objectif de ce TP est la mise en œuvre du codage de canal pour simuler leur capacité de correction
et en cerner les limites. Pour cela, nous utiliserons le logiciel MATLAB avec la ”Communication Toolbox”
et Simulink. Nous allons étudier les deux grandes familles de codeurs de canal :
– Les codes en bloc (linéaires ou cycliques), pour lesquels des blocs de k symboles d’information sont
protégés indépendamment les uns des autres par m symboles de contrôle pour former des mots
codes à n symboles.
– Les codes convolutifs, pour lesquels il n’y a pas de mots codes indépendants, mais où une fenêtre
glissante de largeur m se déplace sur les symboles d’information et permet de coder un symbole
d’information en fonction de m symboles précédents.


\section{Codage de Hamming}

\subsection{Question 1 - Codage}

Pour calculer un mot code à partir de la matrice de contrôle de parité $\mathbf{H} = [\mathbf{P}^T | \mathbf{I}_{n-k}]$ et d'un bloc d'information $\mathbf{m}$, il nous suffit de calculer la matrice génératrice $\mathbf{G}$ qui est définie par $[\mathbf{I}_k | \mathbf{P}]$. Puis on obtient le mot code $\mathbf{c}$ en multipliant $\mathbf{m}$ et $\mathbf{G}$ (multiplication modulo 2). Voir fonction \textit{hamcode} en annexe \ref{hamcode}. La matrice de contrôle de parité $\mathbf{H}$ est quant à elle calculer via la fonction MATLAB \textit{hammgen}. Cette fonction prend un paramètre $M$ permettant le calcul de la longueur du mot code en suivant la relation $N=2^M-1$. \\

\subsection{Question 2 - Décodage}

Afin de décoder le mot code précédemment créé, éventuellement entaché d'erreurs, nous avons mis en place une fonction \textit{hamdecode} (\textit{cf.} \ref{hamdecode}) prenant en paramètre le mot code et la matrice de contrôle de parité.
Pour cela, nous commençons par calculer le syndrome $s = r\times H^T$ ($r$ étant le mot encodé), permettant de retrouver les éventuelles erreurs glissées dans le mot encodé. En effet, si $s$ est différent de 0, alors il existe une erreur dans le mot encodé. Cette erreur se retrouve facilement une fois la table des syndromes calculée.
Ici, nous ajoutons trois bits de redondance au mot à encoder, il y aura donc $2^3$ syndromes possibles. Pour chacun de ces syndromes, on cherche si il existe une configuration $e$ de 1 erreur, puis de 2 erreurs, puis de 3... telle que $s = e \times H^t$. Pour $s=0$, $e=0$ : il n'existe pas d'erreur. Ainsi, nous pouvons détecter des erreurs dans $2^3 - 1$ mots encodés. Une fonction MATLAB permet de calculer directement la table des syndromes en lui passant la matrice de contrôle de parité : \textit{syndtable}. Sur chaque ligne de la matrice retournée par \textit{syndtable} apparait $e$, correspondant à la position du ou des bits d'erreur. Une opération logique XOR permet donc de corriger l'erreur dans le mot à décoder et de retourner le bon mot code. Nous avons fait le test en insérant une erreur :
\begin{alltt}
>> H = hammgen(3); % creation de la matrice de controle de parite
>> c = hamcode([1 0 1 0], H) % encodage du mot 0101
c =
     0     0     1   |   \textit{1     0     1     0}
		
% decodage avec inclusion d'une erreur
>> cdecode = hamdecode( xor([0 1 0 0 0 0 0], c), H)
cdecode =
     0     0     1   |   \textit{1     0     1     0}
\end{alltt}

Nous observons que l'erreur est corrigé.

\section{Codage BCH}

\section{Codes convolutifs}

\newpage

\section{Conclusion}

\clearpage

%
% ANNEXE
%
\appendix

\section{Codes source MATLAB}

\subsection{Code de Hamming : Codage}\label{hamcode}

\FSource{../hamcode.m}

\newpage

\subsection{Code de Hamming : Décodage}\label{hamdecode}

\FSource{../hamdecode.m}

\end{document}