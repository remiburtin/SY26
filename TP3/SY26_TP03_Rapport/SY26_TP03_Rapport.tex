\documentclass[a4paper, 12pt]{article}
\usepackage[utf8]{inputenc}  % Unicode
\usepackage[french]{babel}
\usepackage{geometry}
\usepackage{graphics}
\usepackage[pdftex]{graphicx, color}
\usepackage{pdfpages}
\DeclareGraphicsExtensions{.jpg,.png}
\pdfoutput=1
\usepackage[pdftex,
	bookmarks = true,           % Signets
	bookmarksnumbered = true,   % Signets numerotes
	pdfstartview = FitV,        % La page prend toute la hauteur
	colorlinks=true,
	citecolor=black,urlcolor=blue,linkcolor=black,
	pdfauthor={Auteur},
	pdftitle={Titre},
 	pdfsubject={Sujet},
%	pdfkeywords={},	% Besoin de keywords ?
	plainpages=false,
	pdfpagelabels,
	breaklinks=true,
   	hyperindex,
	linktocpage=true	% pour colorier seulement le numéros dans la TOC	
]{hyperref}
\usepackage{float}
\usepackage{listings}

\newcommand{\HRule}{\rule{\linewidth}{0.5mm}}

\newcommand*\styleC{\fontsize{9}{10pt}\selectfont }
\newcommand*\styleD{\fontsize{9}{10pt}\usefont{OT1}{pag}{m}{n}\selectfont }

\makeatletter
% on fixe le langage utilisé
\lstset{language=matlab}
\edef\Motscle{emph={\lst@keywords}}
\expandafter\lstset\expandafter{%
  \Motscle}
\makeatother


\definecolor{Ggris}{rgb}{0.45,0.48,0.45}

\lstset{emphstyle=\ttfamily\color{blue}, % les mots réservés de matlab en bleu
basicstyle=\ttfamily\styleC, % 
keywordstyle=\ttfamily,
commentstyle=\color{Ggris}\styleD, % \styleD commentaire en gris
numberstyle=\tiny\color{black},
numbers=left,
numbersep=10pt,
lineskip=0.7pt,
showstringspaces=false}
%  % inclure le fichier source
\newcommand{\FSource}[1]{%
\lstinputlisting[texcl=true]{#1}
}

%%%%%%%%%
\textwidth=15cm
\textheight=21cm
%\hoffset=-2.5cm
\tolerance=9000
\hbadness=9000
\pretolerance=2500


\begin{document}
%\rmfamily

\begin{titlepage}
\begin{center}



\textsc{\Large Rapport de TP - SY26}\\[0.5cm]
\vspace{4cm}
% Title
\HRule \\[0.4cm]
{ \huge \bfseries TP04 - La compression JPEG \\[0.4cm] }

\HRule \\[1.5cm]

% Author and supervisor
\begin{minipage}{0.4\textwidth}
\begin{flushleft} \large
R\'emi \textsc{Burtin}
\end{flushleft}
\end{minipage}
\begin{minipage}{0.4\textwidth}
\begin{flushright} \large
Cyril \textsc{Fougeray}
\end{flushright}
\end{minipage}

\vspace{4cm}

{\large \today}



\vfill
% Bottom
\includegraphics[width=0.25\textwidth]{logo.jpg}\\[0.5cm]

\textsc{\LARGE Universit\'{e} de Technologie de Compi\`{e}gne}\\[1.5cm]


\end{center}
\end{titlepage}


%\begin{abstract} 
%\end{abstract} 

%{\bf Keywords:} \newline


\clearpage

\section{Introduction}
L'objectif de ce TP est de programmer l'algorithme itératif de Lloyd-Max pour la conception d'un quantificateur scalaire optimal.

\section{Quantification uniforme}

\subsection{Distorsion normalisée et rapport signal bruit}

Ce premier script nous permet de calculer la distorsion normalisée (NMSE) ainsi que le rapport signal/bruit (SNR).

Pour cela, nous utilisons la fonction \textit{lloyds} qui permet d'encoder un vecteur (ici une image en vecteur ligne) selon Lloyd-Max avec un nombre précis de partitions (correspondant aux nombre de valeurs de reconstruction). Cette fonction \textit{lloyds} retourne la distorsion (non normalisée). Afin de calculer la distorsion normalisée, nous devons diviser cette distorsion $D$ par la variance du vecteur passé à la fonction :
		\[NMSE = \frac{D}{\sigma_X^2} 
	\]


\subsection{Résultats}

Après avoir exécuté la fonction, nous nous rendons compte des résultats : 

\begin{verbatim}
>> [nmse snr] = disto_1_1('lena.bmp', 2)
nmse = 0.3009
snr = 5.2157

>> [nmse snr] = disto_1_1('lena.bmp', 8)
nmse = 0.0190
snr = 17.2149

>> [nmse snr] = disto_1_1('lena.bmp', 256)
nmse = 0
snr = Inf
\end{verbatim}

Lors de la première exécution, nous choisissons d'encoder l'image avec 2 valeurs différentes (correspondant à 2 niveaux de gris dans notre cas). La distorsion est donc importante, et maximale (il n'est pas possible et inutile d'encoder l'image avec 1 seule valeur car aucune information n'est conservée dans ce cas).

Évidemment, dans le cas où nous encodons l'image avec 256 niveaux de gris (telle que l'image originale), aucune distorsion n'est présente.


\section{Quantification scalaire optimale de Lloyd-Max}

\subsection{Mise en oeuvre}



\subsection{Tests sur les différentes images}

\begin{verbatim}
>> LOG
\end{verbatim} \\

\newpage

\section{Conclusion}


\clearpage
\appendix

\section{Codes source MATLAB}
\subsection{NMSE et SNR}\label{nmse_snr_code}

\FSource{../disto_1_1.m}


\end{document}