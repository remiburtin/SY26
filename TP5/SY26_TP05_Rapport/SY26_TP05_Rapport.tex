\documentclass[a4paper, 12pt]{article}
\usepackage[utf8]{inputenc}  % Unicode
\usepackage[french]{babel}
\usepackage{geometry}
\usepackage{graphics}
\usepackage[pdftex]{graphicx, color}
\usepackage{pdfpages}
\DeclareGraphicsExtensions{.jpg,.png}
\pdfoutput=1
\usepackage[pdftex,
	bookmarks = true,           % Signets
	bookmarksnumbered = true,   % Signets numerotes
	pdfstartview = FitV,        % La page prend toute la hauteur
	colorlinks=true,
	citecolor=black,urlcolor=blue,linkcolor=black,
	pdfauthor={Auteur},
	pdftitle={Titre},
 	pdfsubject={Sujet},
%	pdfkeywords={},	% Besoin de keywords ?
	plainpages=false,
	pdfpagelabels,
	breaklinks=true,
   	hyperindex,
	linktocpage=true	% pour colorier seulement le numéros dans la TOC	
]{hyperref}
\usepackage{float}
\usepackage{listings}
\usepackage{alltt}
\usepackage{amsmath}
\renewcommand{\ttdefault}{txtt}

\lstset{basicstyle=\ttfamily,
escapeinside={||},
mathescape=true}
\setcounter{secnumdepth}{3}
\newcommand{\HRule}{\rule{\linewidth}{0.5mm}}

\newcommand*\styleC{\fontsize{9}{10pt}\selectfont }
\newcommand*\styleD{\fontsize{9}{10pt}\usefont{OT1}{pag}{m}{n}\selectfont }

\makeatletter
% on fixe le langage utilisé
\lstset{language=matlab}
\edef\Motscle{emph={\lst@keywords}}
\expandafter\lstset\expandafter{
}
\makeatother



\definecolor{Ggris}{rgb}{0.45,0.48,0.45}

\lstset{emphstyle=\ttfamily\color{blue}, % les mots réservés de matlab en bleu
basicstyle=\ttfamily\styleC, % 
keywordstyle=\ttfamily,
commentstyle=\color{Ggris}\styleD, % \styleD commentaire en gris
numberstyle=\tiny\color{black},
numbers=left,
numbersep=10pt,
lineskip=0.7pt,
showstringspaces=false}
%  % inclure le fichier source
\newcommand{\FSource}[1]{%
\lstinputlisting[texcl=true]{#1}
}

%%%%%%%%%
\textwidth=15cm
\textheight=21cm
%\hoffset=-2.5cm
\tolerance=9000
\hbadness=9000
\pretolerance=2500


\begin{document}
%\rmfamily

\begin{titlepage}
\begin{center}



\textsc{\Large Rapport de TP - SY26}\\[0.5cm]
\vspace{4cm}
% Title
\HRule \\[0.4cm]
{ \huge \bfseries TP04 - La compression JPEG \\[0.4cm] }

\HRule \\[1.5cm]

% Author and supervisor
\begin{minipage}{0.4\textwidth}
\begin{flushleft} \large
R\'emi \textsc{Burtin}
\end{flushleft}
\end{minipage}
\begin{minipage}{0.4\textwidth}
\begin{flushright} \large
Cyril \textsc{Fougeray}
\end{flushright}
\end{minipage}

\vspace{4cm}

{\large \today}



\vfill
% Bottom
\includegraphics[width=0.25\textwidth]{logo.jpg}\\[0.5cm]

\textsc{\LARGE Universit\'{e} de Technologie de Compi\`{e}gne}\\[1.5cm]


\end{center}
\end{titlepage}


%\begin{abstract} 
%\end{abstract} 

%{\bf Keywords:} \newline


\clearpage

\section{Introduction}

Le but de ce TP est de mettre en oeuvre la technique du block matching utilisée dans certains algorithmes de compression vidéo.

\section{Mise en oeuvre du block matching}

\subsection{Padding de l'image}

La division de l'image en bloc implique que la largeur et la hauteur de l'image soient respectivement multiples de la largeur et de la hauteur des blocs. Si ce n'est pas le cas, nous complétons avec des zéros (pixels noirs) grâce à la fonction padarray. Pour calculer le nombre de pixels à rajouter en largeur on utilise la formule suivante : \\

\begin{center}
	$(M - (largeur(image) \mod M)) \mod M$ \\
\end{center}
avec M largeur d'un bloc.

\begin{figure}[H]
	\centering
		\includegraphics[height=6cm]{../garden_padding.jpg}
	\caption{Image, issue de la video garden, que l'on a voulu diviser en bloc de 7x7. L'image faisant 352x240, il a fallu rajouter 5 pixels noirs en largeur et 5 en hauteur.}
	\label{fig:padding}
\end{figure}

\subsection{Calcul de la fenêtre de recherche}

\subsection{Recherche du meilleur bloc}

\section{Résultats}

\newpage

\section{Conclusion}


\clearpage

%
% ANNEXE
%
\appendix

\section{Codes source MATLAB}

\subsection{Calcul du MSD}\label{msd_code}

\FSource{../compute_msd.m}

\newpage

\subsection{Calcul de la fenêtre de recherche}\label{search_window}

\FSource{../search_window.m}

\newpage

\subsection{Recherche d'un bloc dans l'image de référence}\label{block_search}

\FSource{../block_matching.m}

\newpage

\subsection{Block Matching}\label{block_matching}

\FSource{../block_matching_encode.m}


\end{document}